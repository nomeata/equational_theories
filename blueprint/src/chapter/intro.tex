\chapter{Basic theory of magmas}

\begin{definition}[Magma]\label{magma-def}\lean{Magma}\leanok A \emph{magma} is a set $G$ equipped with a binary operation $\circ: G \times G \to G$.  A \emph{homomorphism} $\varphi : G \to H$ between two magmas is a map such that $\varphi(x \circ y) = \varphi(x) \circ \varphi(y)$ for all $x,y \in G$.  An \emph{isomorphism} is an invertible homomorphism.
\end{definition}

Groups, semi-groups, and monoids are familiar examples of magmas.  However, in general we do not expect magmas to have any associative properties.

A magma is called \emph{empty} if it has cardinality zero, \emph{singleton} if it has cardinality one, and \emph{non-trivial} otherwise.

The number of magma structures on a set $G$ of cardinality $n$ is of course $n^{n^2}$, which is \footnote{All sequences start from $n=0$ unless otherwise specified.}
$$ 1, 1, 16, 19683, 4294967296, 298023223876953125, \dots$$
(\href{OEIS A002489}{https://oeis.org/A002489}).
Up to isomorphism, the number of finite magmas of cardinality $n$ up to isomorphism is the slightly slower growing sequence
$$ 1, 1, 10, 3330, 178981952, 2483527537094825, 14325590003318891522275680, \dots$$
(\href{OEIS A001329}{https://oeis.org/A001329}).

\begin{definition}[Free Magma]\label{free-magma-def}\lean{FreeMagma}\leanok\uses{magma-def} The \emph{free magma} $M_X$ generated by a set $X$ (which we call an \emph{alphabet}) is the set of all finite formal expressions built from elements of $X$ and the operation $\circ$.  An element of $M_X$ will be called a \emph{word} with alphabet $X$.  The \emph{order} of a word is the number of $\circ$ symbols needed to generate the word.  Thus for instance $X$ is precisely the set of words of order $1$ in $M_X$.
\end{definition}

For sake of concreteness, we will take the alphabet $X$ to default to the natural numbers $\N$ if not otherwise specified.

For instance, if $X = \{0,1\}$, then $M_X$ would consist of the following words:
\begin{itemize}
  \item $0$, $1$ (the words of order $0$);
  \item $0 \circ 0$, $0 \circ 1$, $1 \circ 0$, $1 \circ 1$ (the words of order $1$);
  \item $0 \circ (0 \circ 0)$, $0 \circ (0 \circ 1)$, $0 \circ (1 \circ 0)$, $0 \circ (1 \circ 1)$, $1 \circ (0 \circ 0)$, $1 \circ (0 \circ 1)$, $1 \circ (1 \circ 0)$, $1 \circ (1 \circ 1)$, $(0 \circ 0) \circ 0$, $(0 \circ 0) \circ 1$, $(0 \circ 1) \circ 0$, $(0 \circ 1) \circ 1$, $(1 \circ 0) \circ 0$, $(1 \circ 0) \circ 1$, $(1 \circ 1) \circ 0$, $(1 \circ 1) \circ 1$ (the words of order $2$);
  \item etc.
\end{itemize}

\begin{lemma}  For a finite alphabet $X$, the number of words of order $n$ is $C_n |X|^{n+1}$, where $C_n$ is the $n^{\mathrm{th}}$ Catalan number and $X$ is the cardinality of $X$.
\end{lemma}

\begin{proof} Follows from standard properties of Catalan numbers.
\end{proof}

The first few Catalan numbers are
$$ 1, 1, 2, 5, 14, 42, 132, \dots$$
(\href{OEIS A000108}{https://oeis.org/A000108}).


\begin{definition}[Induced homomorphism]\label{induced-def}\uses{free-magma-def}  Given a function $f: X \to G$ from an alphabet $X$ to a magma $G$, the \emph{induced homomorphism} $\varphi_f: M_X \to G$ is the unique extension of $f$ to a magma homomorphism.  Similarly, if $\pi \colon X \to Y$ is a function, we write $\pi_* \colon M_X \to M_Y$ for the unique extension of $\pi$ to a magma homomorphism.
\end{definition}

For instance, if $f : \{0,1\} \to G$ maps $0,1$ to $x,y$ respectively, then
$$ \varphi_f(0 \circ 1) = x \circ y$$
$$ \varphi_f(1 \circ (0 \circ 1)) = y \circ (x \circ y)$$
and so forth.  If $\pi \colon \N \to \N$ is the map $\pi(n) := n+1$, then
$$ \pi_*(0 \circ 1) = 1 \circ 2$$
$$ \pi_*(1 \circ (0 \circ 1)) = 2 \circ (1 \circ 2)$$
and so forth.

\begin{definition}[Law]\label{law-def}\uses{induced-def}  Let $X$ be a set. A \emph{law} with alphabet $X$ is a formal expression of the form $w == w'$, where $w, w' \in M_X$ are words with alphabet $X$ (thus one can identify laws with alphabet $X$ with elements of $M_X \times M_X$).  A magma $G$ \emph{satisfies} the law $w == w'$ if we have $\varphi_f( w ) = \varphi_f ( w' )$ for all $f: X \to G$, in which case we write $G \models w == w'$.
\end{definition}

Thus, for instance, the commutative law
\begin{equation}\label{comm-law}
  0 \circ 1 == 1 \circ 0
\end{equation}
is satisfied by a magma $G$ if and only if
\begin{equation}\label{comm-law-2}
 x \circ y = y \circ x
\end{equation}
for all $x, y \in G$.  We refer to \eqref{comm-law-2} as the \emph{equation} associated to the law \eqref{comm-law}.  One can think of equations as the ``semantic'' intrepretation of a ``syntactic'' law.  However, we shall often abuse notation and a law with its associated equation thus we shall (somewhat carelessly) also refer to \eqref{comm-law-2} as ``the commutative law'' (rather than ``the commutative equation'').

\begin{lemma}[Pushforward]\label{push}\uses{law-def}  Let $w == w'$ be a law with some alphabet $X$, $G$ be a magma, and $\pi: X \to Y$ be a function.  If $G \models w == w'$, then $G \models \pi_*(w) == \pi_*(w')$.  In particular, if $\pi$ is a bijection, the statements If $G \models w == w'$, then $G \models \pi_*(w) == \pi_*(w')$ are equivalent.
\end{lemma}

If $\pi$ is a bijection, we will call $\pi_*(w) == \pi_*(w')$ a \emph{relabeling} of the law $w == w'$.  Thus for instance
$$ 5 \circ 7 == 7 \circ 5$$
is a relabeling of the commutative law \eqref{comm-law}.  By the above lemma, relabeling does not affect whether a given magna satisfies a given law.

\begin{proof}  Trivial.
\end{proof}

\begin{lemma}[Equivalence]\label{equiv}\uses{law-def}  Let $G$ be a magma and $X$ be an alphabet.  Then the relation $G \models w == w'$ is an equivalence relation on $M_X$.
\end{lemma}

\begin{proof}  Trivial.
\end{proof}

Define the total order of a law $w == w'$ to be the sum of the orders of $w$ and $w'$.

\begin{lemma}[Counting laws up to relabeling]\label{law-count}\uses{push}  Up to relabeling, the number of laws $w == w'$ of total order $n$ is $C_{n+1} B_{n+2}$.
\end{lemma}

\begin{proof} Follows from the properties of Catalan and Bell numbers.
\end{proof}

The first few Bell numbers (starting from $n=0$) are
$$ 1, 1, 2, 5, 15, 52, 203, \dots$$
(\href{OEIS A000110}{https://oeis.org/A000110}).

The sequence in Lemma \ref{law-count} is
$$ 2, 10, 75, 728, 8526, 115764, \dots$$
(\href{OEIS A289679}{https://oeis.org/A289679}).

Now we would also like to count laws up to relabeling and symmetry.

\begin{lemma}[Counting laws up to relabeling and symmetry]\label{law-count-sym}\uses{push} Up to relabeling and symmetry, the number of laws $w == w'$ of total order $n$ is
$$ C_{n+1} B_{n+2}/2$$
when $n$ is odd, and
$$ (C_{n+1} B_{n+2} + C_{n/2} (2D_{n+2} - B_{n+2}))/2$$
when $n$ is even, where $D_n$ is the number of partitions of $[n]$ up to reflection.
\end{lemma}

\begin{proof} Elementary counting.
\end{proof}

The sequence $D_n$ is (starting from $n=0$)
$$ 1, 1, 2, 4, 11, 32, 117, \dots$$
(\href{OEIS A103293}{https://oeis.org/A103293}), and the sequence in Lemma \ref{law-count-sym} is (starting from $n=0$)
$$ 2, 5, 41, 364, 4294, 57882, 888440, \dots$$
{\bf check this matches the formula}

We can also identify all laws of the form $w==w$ with the trivial law $0==0$.  The number of such laws of total order $n$ is zero if $n$ is odd, and $C_{n/2} B_{n/2+1}$ if $n$ is even.  We conclude:

\begin{lemma}[Counting laws up to relabeling, symmetry, and triviality]  Up to relabeling, symmetry, and triviality, the number of laws of total order $n$ is
$$ C_{n+1} B_{n+2}/2$$
if $n$ is odd, $2$ if $n = 0$, and
$$ (C_{n+1} B_{n+2} + C_{n/2} (2D_{n+2} - B_{n+2}))/2 - C_{n/2} B_{n/2+1}$$
if $n \geq 2$ is even.
\end{lemma}

\begin{proof} Routine counting.
\end{proof}

This sequence is
$$2, 5, 39, 364, 4284, 57882, 888365, \dots$$
{\bf check this matches the formula}.

In particular, up to relabeling, symmetry, and triviality, there are exactly $4694$ laws of total order at most $4$.  A list can be found \href{https://github.com/teorth/equational_theories/blob/main/data/equations.txt}{here}.  A script for generating them may be found \href{https://github.com/teorth/equational_theories/blob/main/scripts/generate_eqs_list.py}{here}.  The list is sorted by the total number of operations, then by the number of operations on the LHS. Within each such class we define an order on expressions by variable $<$ operation, and lexical order on variables.
